\documentclass[onlymath, nologo]{beamer}

\usepackage{amsmath, amsfonts}
\usepackage{tikz}
\usetikzlibrary{decorations.pathreplacing,calc}
\usepackage{tabulary}
\usepackage{array}
{\renewcommand{\arraystretch}{2}%
\usepackage{mathtools}
\setbeamertemplate{caption}{\raggedright\insertcaption\par}
\usepackage{multimedia}
\usepackage[version=4]{mhchem}
\usepackage[theorems, skins]{tcolorbox}
% The following three lines are used to split an enumerate
% across columns.
\newcounter{savedenum}
\newcommand*{\saveenum}{\setcounter{savedenum}{\theenumi}}
\newcommand*{\resume}{\setcounter{enumi}{\thesavedenum}}

% For getting a brace to span items in an itemize environment
\newcommand{\tikzmark}[1]{\tikz[overlay,remember picture] \node (#1) {};}

\graphicspath{{../figs/}}

\definecolor{UTBlue}{RGB}{94,156,174}
\definecolor{HCrimson}{RGB}{165,28,48}
\definecolor{HInk}{RGB}{30,30,30}
\definecolor{HMortar}{RGB}{140,129,121}
\definecolor{HParchment}{RGB}{243,243,241}
\definecolor{HSlate}{RGB}{137,150,160}
\definecolor{HShade}{RGB}{186,197,198}
\tcbset{colback=HParchment,colframe=HCrimson}

\usepackage{hyperref}
\newcommand\myshade{85}
\colorlet{mylinkcolor}{HCrimson}
\colorlet{mycitecolor}{UTBlue}
\colorlet{myurlcolor}{HCrimson}

\hypersetup{
  linkcolor  = mylinkcolor!\myshade!black,
  citecolor  = mycitecolor!\myshade!black,
  urlcolor   = myurlcolor!\myshade!black,
  colorlinks = true,
}

\usetheme[white]{Harvard}

\author[]{Instructor:  David Sondak \\ TFs:  Charles Liu, Eric Wu and Kevin Wu}
\title{CS207: Systems Development for Computational Science}
\subtitle{Version Control:  Big ideas and \texttt{Git}}

\institute{Harvard University \\ 
           Institute for Applied Computational Science}
\date{\today}

\begin{document}

\bgroup
\makeatletter
\setbeamertemplate{footline}{}
\makeatother

  \begin{frame}
    \titlepage
  \end{frame}
  \egroup
  
  \setcounter{framenumber}{0}

  \begin{frame}{Today's Plan}
    A large portion of today's lecture is based upon a version control lecture given by 
    Chris Simmons at UT-Austin in his course \textit{Tools and Techniques of Computational Science}. \\[1.0em]
    The topics for today are:
    \begin{itemize}
      \item Lecture workflow
      \item Version Control --- Big picture 
      \item Types of version control 
      \item \texttt{Git}
    \end{itemize}
  \end{frame}

  \begin{frame}{Lecture Workflow}
    At the beginning of each lecture, do the following:
    \begin{enumerate}
      \item Open a terminal 
      \item \texttt{cd cs207\_firstname\_lastname}
      \item \texttt{git pull upstream master} 
      \item \texttt{git add .} 
      \item \texttt{git commit -m "Updated cs207 notes for Lecture n" files\_to\_add}
      \item \texttt{git push}
    \end{enumerate}
    We need to talk about all of this ($\Uparrow$): why we do it and what it means.
  \end{frame}

  \begin{frame}{The Multiple Roles of a Computational Scientist}
    \begin{itemize}
      \item Computational scientists are expected to be:
        \begin{itemize}
          \item Physicist --- addressed in curriculum  \\[0.5em]
          \item Numerical analyst --- addressed in curriculum  \\[0.5em]
          \item Software developer --- not usually addressed
        \end{itemize}
      \item Software engineering is vital
        \begin{itemize}
          \item Understandable code
        \end{itemize}
    \end{itemize}
    \begin{tcolorbox}[title=Software engineering gap, arc is angular]
      Software engineering is not part of the formal curriculum and therefore 
      most scientific software developed in academia is unverified.  Furthermore, 
      the software developed is unable to ever be verified without significant 
      time investment.
    \end{tcolorbox}
  \end{frame}

  \begin{frame}{Version Control}
    \begin{itemize}
      \item Minimum guidlines --- Actually using version control is the first step \\[1.0em]
      \item Ideal usage:
        \begin{itemize}
          \item Put \textbf{everything} under version control \\[0.5em]
          \item Consider putting parts of your home directory under version control \\[0.5em]
          \item Use a consistent project structure and naming convention \\[0.5em]
          \item Commit often and in logical chunks \\[0.5em]
          \item Write meaningful commit messages \\[0.5em]
          \item Do all file operations in the version control system \\[0.5em]
          \item Set up change notifications if working with multiple people
        \end{itemize}
    \end{itemize}
  \end{frame}

  \begin{frame}{Source Control and Versioning}
    \begin{itemize}
      \item Why bother? \\[1.0em]
      \item Codes evolve over time \\[0.5em]
        \begin{itemize}
          \item Sometimes bugs creep in (by you or others) \\[0.5em]
          \item Sometimes the old way was right  \\[0.5em]
          \item Sometimes it's nice to look back at the evolution \\[1.0em]
        \end{itemize}
      \item How can you get back to an old version? \\[0.5em]
        \begin{itemize}
          \item Keep a copy of every version of every file \\[0.5em] 
            \begin{itemize}
              \item Disk is cheap, but this could get out of hand quickly  \\[0.25em]
              \item \textbf{Huge} pain to maintain \\[0.5em]
            \end{itemize}
          \item Use a tool
        \end{itemize}
    \end{itemize}
  \end{frame}

  \begin{frame}{Some Example Tools}
    \begin{itemize}
      \item Free
      \begin{itemize}
        \item RCS --- Revision Control System \tikzmark{topbraceC} 
        \item CVS --- Concurrent Versions System \tikzmark{rightC}
        \item SVN --- Subversion  \tikzmark{bottombraceC} 
        \item Git --- \url{https://git-scm.com/} \tikzmark{topbraceD}
        \item Mercurial --- \url{https://www.mercurial-scm.org/} \tikzmark{bottombraceD} \tikzmark{rightD}
      \end{itemize}
      % From https://tex.stackexchange.com/questions/12391/large-braces-over-several-items-in-an-itemize-with-text-by-the-brace
      \begin{tikzpicture}[overlay, remember picture]
        \draw [decoration={brace,amplitude=0.5em},decorate,ultra thick,gray]
              ($(rightC)!(topbraceC.north)!($(rightC)-(0,1)$) + (0,0.8em)$) --  node[right=0.6em] {Centralized} ($(rightC)!(bottombraceC.south)!($(rightC)-(0,1)$)$);
      \end{tikzpicture}
      \begin{tikzpicture}[overlay, remember picture]
        \draw [decoration={brace,amplitude=0.5em},decorate,ultra thick,gray]
              ($(rightD)!(topbraceD.north)!($(rightD)-(0,1)$) + (0,0.8em)$) --  node[right=0.6em] {Distributed} ($(rightD)!(bottombraceD.south)!($(rightD)-(0,1)$)$);
      \end{tikzpicture}
    \end{itemize}
    \begin{itemize}
      \item Commercial
      \begin{itemize}
        \item MS Visual Studio Team System
        \item IBM Rational Software:  Clearcase
        \item AccuRev
        \item MKS Integrity
      \end{itemize}
    \end{itemize}
  \end{frame}

  \begin{frame}{Comments on Centralized Source Control}
    \begin{itemize}
      \item A central repository holds the files in both of the following models
      \begin{itemize}
        \item This means a specific computer is required with some disk space
        \item It should be backed up!
      \end{itemize}
    \end{itemize}
    \begin{columns}[T]
      \begin{column}{0.5\textwidth}
        \begin{enumerate}
          \item Read-only Local Workspaces and Locks
          \begin{itemize}
            \item Every developer has a read-only local copy of the source files 
            \item Individual files are checked-out as needed and locked in the repo in 
                  order to gain write access 
            \item Unlocking the file commits the changes to the repo and makes the file 
                  read-only again
          \end{itemize}
          \saveenum
        \end{enumerate}
      \end{column}
      \begin{column}{0.5\textwidth}
        \begin{enumerate}
          \resume
          \item Read / Write Local Workspaces and Merging
          \begin{itemize}
            \item Every developer has a local copy of the source files
            \item Everybody can read and write files in their local copy
            \item Conflicts between simultaneous edits handled with merging algorithms 
                  or manually when files are synced against the repo or committed to it 
            \item CVS and Subverions behave this way
          \end{itemize}
        \end{enumerate}
      \end{column}
    \end{columns}
  \end{frame}

  \begin{frame}{CVS --- Concurrent Versions System}
    \begin{itemize}
      \item Started with some shell scripts in 1986 \\[0.5em]
      \item Recoded in 1989 \\[0.5em]
      \item Evolving ever since (mostly unchanging now) \\[0.5em]
      \item Uses read / write local workspaces and merging \\[0.5em]
      \item Only stores differences between versions \\[0.25em]
      \begin{itemize}
        \item Saves space
        \item Basically uses \texttt{diff(1)} and \texttt{diff3(1)}
      \end{itemize}
      \item Works with local repositories or over the network with rsh / ssh
    \end{itemize}
  \end{frame}

  \begin{frame}{Subversion}
    Subversion is a functional superset of CVS (if you learned CVS previously, you can 
    also function in Subversion)
    \begin{itemize}
      \item Began initial development in 2000 as a replacement for CVS \\[0.5em] 
      \item Also interacts with local copies \\[0.5em]
      \item Includes directory versioning (rename and moves) \\[0.5em]
      \item Truly atomic commits \\[0.25em]
      \begin{itemize}
        \item i.e. interrupted commit operations do not cause repository inconsistency 
              or corruption
      \end{itemize}
      \item File meta-data \\[0.5em]
      \item True client-server model \\[0.5em] 
      \item Cross-platform, open-source
    \end{itemize}
  \end{frame}

  \begin{frame}{Getting Started with \texttt{Git}}
    There are \textbf{many} \texttt{Git} tutorials:
    \begin{itemize}
      \item \url{https://stackoverflow.com/questions/315911/git-for-beginners-the-definitive-practical-guide}%#323764} 
      \item \url{https://bitbucket.org/} 
      \item \url{https://github.com/}
      \item \hspace{2.0em} $\vdots$
      \item Others on the course \href{https://iacs-cs-207.github.io/cs207-F17/resources.html}{Resources} page
    \end{itemize}
  \end{frame}

\end{document}
